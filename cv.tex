%!TEX TS-program = xelatex
\documentclass[]{friggeri-cv}
\usepackage{wasysym}
\usepackage{marvosym}
\usepackage[utf8]{inputenc} 

\addbibresource{bibliography.bib}

\begin{document}
\header{Alexandre}{Bléron}
       {Ph.D. candidate in computer graphics}


% In the aside, each new line forces a line break
\begin{aside}
  \section{Contact}
    6 rue des Peupliers
    38400 Saint-Martin-d'Hères
    France
    ~
    \href{mailto:alex.bleron@gmail.com}{alex.bleron@gmail.com}
\phone~+33606482283
  \section{Languages}
    French 
    English (TOEIC: 990)
    Spanish notions
    Japanese notions (currently taking courses)
  \section{Programming}
    C++, C\#, Java, Lua, VHDL, x86 assembly
    OpenGL/GLSL, SFML, Antlr, Qt
    \section{Software}
    Photoshop, Maya, Unity, CMake, Git, MS~Office, \LaTeX
\end{aside}

\section{Interests}

Computer graphics, shader programming, stylized (non-photorealistic) and artist-directed rendering, procedural generation, C++

\section{Education}

\begin{entrylist}
  \entry
    {since 2015}
    {Ph.D. candidate in computer graphics}
    {INRIA/Laboratoire Jean Kuntzmann}
    {Real-time stylized rendering techniques for 3D scenes.\\ Goals:
    \begin{itemize} 	\item Be able to use digital painting effects and techniques for the stylization of animated 3D scenes.
    \item Propose new techniques to increase the range of styles achievable with real-time stylization primitives.
    \end{itemize}
    Keywords: stylized rendering, temporal coherence, artistic control.}
  \entry
    {2012–2015}
    {Master's degree}
    {Grenoble INP - Ensimag}
    {Followed the Master of Science in Informatics at Grenoble programme (MoSIG). Specialization in graphics, computer vision and robotics.}
  \entry
    {2010–2012}
    {Classes Préparatoires aux Grandes Écoles}
    {Clermont-Ferrand}
    {Preparatory courses. Specialization in physics, mathematics and engineering science.}
 
\end{entrylist}

\section{Experience}

\begin{entrylist}
   \entry
    {Feb-Jul 2015}
    {INRIA – Research internship}
    {Grenoble}
    {
    Developed an interactive system for the edition of programmable vector textures, extending the framework proposed by Loi \emph{et al.} (\href{https://hal.inria.fr/hal-01141869}{https://hal.inria.fr/hal-01141869}).
    %	\item Proposed a technique to interactively select arbitrary repeating patterns on a vector texture and generate code snippets to represent and manipulate the patterns in a programmable way.
}
  \entry
    {Jul-Aug 2014}
    {CGG – Internship}
    {Massy}
    {Developed a standalone version of a seismic imaging algorithm (Reverse Time Migration) for profiling.\\
    	Analyzed memory access patterns of the algorithm and its CPU cache behavior. Optimized the implementation for a recent CPU architecture.}
    	
\end{entrylist}

\section{Projects}

\begin{entrylist}
  \entry
    {2012--2015}
    {Ensimag projects}
    {}
    {\vspace{-4mm}\begin{itemize}
    \item Procedural generation of 3D models of fortresses on arbitrary terrains using shape grammars. (\href{https://github.com/ennis/stronghold}{Github link})
    \item Developement of a compiler for a Java-like language
    %\item Developement of a JPEG decoder/encoder
    %\item Design of a CPU running the MIPS instruction set and its peripherals (in VHDL), and implementation on an FPGA. Ported a version of the GCC compiler for the designed CPU. 
    \end{itemize}}

  \entry
    {}
    {Personal C++ projects}
    {}
    {\vspace{-4mm}\begin{itemize}
\item Small rendering engine using a path tracing algorithm. (\href{https://github.com/ennis/path-tracer}{Github link})
\item Lua-scriptable graphics framework on top of OpenGL/GLSL (work in progress) 
\end{itemize}}

\end{entrylist}



%%% This piece of code has been commented by Karol Kozioł due to biblatex errors. 
% 
%\printbibsection{article}{article in peer-reviewed journal}
%\begin{refsection}
%  \nocite{*}
%  \printbibliography[sorting=chronological, type=inproceedings, title={international peer-reviewed conferences/proceedings}, notkeyword={france}, heading=subbibliography]
%\end{refsection}
%\begin{refsection}
%  \nocite{*}
%  \printbibliography[sorting=chronological, type=inproceedings, title={local peer-reviewed conferences/proceedings}, keyword={france}, heading=subbibliography]
%\end{refsection}
%\printbibsection{misc}{other publications}
%\printbibsection{report}{research reports}

\end{document}
